%%\documentclass[onecolumn,preprint,showpacs,preprintnumbers,amsmath,amssymb,floatfix]{elsart}
\documentclass[onecolumn,amsmath,amssymb,floatfix]{elsart}

\usepackage{epsfig}


\begin{document}

%\DeclareGraphicsExtensions{.pdf}
\DeclareGraphicsExtensions{.jpg}

% Elsevier change
\begin{frontmatter}


\title
{
Numerical self-consistent field theory
simulations of confined polydisperse block copolymers.
}

% Elsevier change
% \date{\today}

% Elsevier change, author list

\author{Scott W. Sides\corauthref{cor}},
\ead{swsides@txcorp.com}
\author{Rajeev Kumar}

\corauth[cor]{Corresponding author}



% Elsevier change
\address
{
Oak Ridge National Labs (ORNL) Oak Ridge, TN \\
Tech-X Corporation Boulder, CO 80303
}


\begin{abstract}

An efficient algorithm is presented for numerically evaluating a
self-consistent field theoretic (SCFT) model of block copolymer
structure.
A method is outlined for performing simulations of confined polymer
systems for arbitrary spatial configurations.
This algorithm is implemented on a distributed memory
parallel cluster in order to solve the SCFT equations on large
computational grids.
Simulation results are presented for an $AB$ diblock
copolymer in various confinement geometries.


\end{abstract}

\begin{keyword}
block copolymer \sep self-consistent field theory \sep polymer melts \sep parallel computing
\PACS 61.25.Hq \sep 61.43.Bn \sep 64.75.+g \sep 64.60.Cn \sep 64.70.Nd
\end{keyword}

% Elsevier change
% \pacs
% {
% 61.25.Hq, 61.43.Bn, 64.75.+g, 64.60.Cn, 64.70.Nd
% }

% 61.25.Hq  Macromolecular and polymer solutions; polymer melts; swelling
% 61.43.Bn Structural modeling: serial-addition models, computer simulation
% 64.75.+g Solubility, segregation, and mixing; phase separation
% 64.60.Cn Order-isorder transformations; statistical mechanics of model systems
% 64.70.Nd Structural transitions in nanoscale materials

% Elsevier change
% \maketitle


% Elsevier change
\end{frontmatter}


%%%%%%%%%%%%%%%%%%%%%%%%%%%%%%%%%%%%%%%%%%%%%%%%%%%%%%%%%
\section{Introduction}
\label{sec_intro}
%%%%%%%%%%%%%%%%%%%%%%%%%%%%%%%%%%%%%%%%%%%%%%%%%%%%%%%%%



%
 \begin{eqnarray}
 % Elsevier change XXX ALL \cal GONE
 H =
  \frac{1}{4 R_{{\rm g}0}^2 } \sum^{n}_{\alpha = 1}
           \int_0^1 ds \left ( \frac{ d {\vec r}_{\alpha}(s)}{ds} \right )^{2} 
 \nonumber
 & + & \rho_0^{-1} \int d{\vec r} \ \chi_{AB} \ {\hat \rho_A}({\vec r}) {\hat \rho_B}({\vec r}) \\
 & + & \rho_0^{-1} \int d{\vec r} \ \chi_{WA} \    \varrho_W ({\vec r}) {\hat \rho_A}({\vec r}) \\
 \nonumber
 & + & \rho_0^{-1} \int d{\vec r} \ \chi_{WB} \    \varrho_W ({\vec r}) {\hat \rho_B}({\vec r})  .
 \end{eqnarray}
%
The first term in $H$ is the free-energy contribution from
the so-called ``Gaussian thread'' \cite{doi_book} model.
This term can be
derived as the continuous limit of a chain model consisting of
coarse-grained, spherical beads connected by harmonic springs,
where $R_{{\rm g}0}$ is the radius of gyration for the unperturbed
Gaussian chain, $n$ is the number of chains and ${\vec
r}_{\alpha}(s)$ is a space curve parameterized by the chain
contour variable $s$ which describes the configuration of the
$\alpha^{th}$ chain.
The second term is obtained from a Flory-type
model of the interaction energy between chemically distinct
monomer species $A$ and $B$, which is parameterized by $\chi_{AB}$.
The last two terms are also obtained from a Flory-type
model of the interaction energy between the $A$/$B$ monomer species
and the particles in the wall and is parameterized by $\chi_{WA}$/$\chi_{WB}$
respectively.
The total density for chain monomers is $\rho_0$=$n N/V$, where $N$ is the
polymerization index and $V$ is the volume containing a finite amount
of polymer chains (this will be made explicit below).

{\bf
The positions of each wall-monomer can in principle be
described explicitly with a function ${\hat \varrho_W} ({\vec r})$,
as are the monomers within the diblock chains.
However, this is unnecessary because these particles
are forced to be stationary and therefore the entropy due
to the wall-particles does not contribute to the total partition function.
The particle density inside the wall is assumed to be sufficiently large,
such that the continuous density field $\varrho_W ({\vec r})$ is an appropriate
description of the interactions between particles in the wall and monomers
along the polymer chains.
}

For these simulations $\chi_{AB} > 0$, indicating an overall repulsive force between
dissimilar monomers which drives phase separation.
The wall interaction parameters $\chi_{WA}$ and $\chi_{WB}$ determine the 
a tendency for the polymer chains to either wet or be excluded from the confining surface.
{\bf 
The relative sizes of the $\chi$ parameters determine the effective attraction
or replusion for a monomer species to a wall species.
For example, if $\chi_{WA} < \chi_{WB}$ then $A$ monomers preferentially wet the
surface wall. If $\chi_{WB} < \chi_{WA}$ then $B$ monomers preferentially wet the
surface wall and if $\chi_{WB}$=$\chi_{WA}$ then the wall if effectively neutral.
The absolute sizes of $\chi_{WA}$, $\chi_{WB}$.... determine how strongly the monomers
are excluded from the confinement walls.
}

Note that while
$\chi$ is known experimentally to depend on temperature, there is
no such explicit dependence in the present theoretical
formulation.
The predictions for the mesoscopic structure of the
block copolymer systems studied in this paper are for particular values of
$\chi$, which depend not only on the specific interactions between
chemical species $A$ and $B$, between $A$/$B$ monomers and the wall,
but on the temperature as well.

The $A$/$B$ monomer density operators and the wall density function are
%
 \begin{eqnarray}
 {\hat \rho_A}({\vec r})
     & = & N \sum^{n}_{\alpha = 1}
           \int_0^f ds \ \delta({\vec r} - {\vec r}_{\alpha}(s)) \\
 {\hat \rho_B}({\vec r})
     & = & N \sum^{n}_{\alpha = 1}
           \int_f^1 ds \ \delta({\vec r} - {\vec r}_{\alpha}(s)) \\
 \varrho_W ({\vec r})
     & \in &  [0,\rho_0] .
 \end{eqnarray}
%
where $f$ is the fraction of the chain consisting of $A$ monomers
and
{\bf
$\varrho_W ({\vec r})$ can
in general be an arbitrary function of position.
}


The partition function for this system can be written as
%(eqn untransformed Z)
%
 \begin{eqnarray}
 \nonumber
 \label{eqn_untransformed_Z}
  & Z & =
      \int \prod^{n}_{\alpha = 1} \tilde{D} {\vec r}_{\alpha} \
      \delta[\rho_0 - \varrho_W ({\vec r}) - {\hat \rho_A} - {\hat \rho_B}] \times \\
  \nonumber
  & \exp &
  \nonumber
  \left \lbrace \rho_0^{-1} \int d{\vec r} \
  [\chi_{AB} \ {\hat \rho_A}({\vec r}) {\hat \rho_B}({\vec r}) +
   \chi_{WA} \    \varrho_W ({\vec r}) {\hat \rho_A}({\vec r}) +
   \chi_{WB} \    \varrho_W ({\vec r}) {\hat \rho_B}({\vec r})
    \right \rbrace
 \end{eqnarray}
%
with
%
 \begin{eqnarray}
  \tilde{D} {\vec r}_{\alpha}
    & = &
    D {\vec r}_{\alpha}
      \exp \left \lbrace
           - \frac{1}{4 R_{{\rm g}0}^2 } \int_0^1
           ds \left ( \frac{ d {\vec r}_{\alpha}(s)}{ds} \right )^{2}
           \right \rbrace
 \end{eqnarray}
where $\int D {\vec r}_\alpha$ denotes a path integral over
all possible conformations of the $\alpha$th chain and the
{\bf
delta function enforces an incompressibility constraint among $A$ monomers,
$B$ monomers and the wall such that the total monomer density is kept
constant.
}
The partition function is now expressed in terms of
monomer density operators which enumerate the positions of the
$A$/$B$ monomers on the $n$ polymer chains.
This expression for
$Z$ may be converted into a field-theoretic description by a
Hubbard-Stratonovich (HS) transformation
by insertion of the functional integral identity
%(eqn functional identity)
%
 \begin{eqnarray}
 \label{eqn_functional_identity}
  1
  & = &
  \int D[\rho] \ \delta[\rho - {\hat \rho} ] \\
  \nonumber
  & = &
  \int D[\rho] D[\omega]
    \exp
    \left \lbrace
     \int d{\vec r} \ i w({\vec r})
       \left [ \rho({\vec r}) - {\hat \rho}({\vec r}) \right ]
     \right \rbrace
 \end{eqnarray}
%
where the chemical potential field $w({\vec r})$ has been introduced through
the exponential representation of the delta functional
$\delta[\rho - {\hat \rho} ]$.
Using the identity in Eqn.~\ref{eqn_functional_identity} for both $A$ and $B$ density
operators and substituting the exponential form of the incompressibility constraint
gives
%(eqn transformed Z)
%(eqn free energy 1)
%(eqn single chain Q)
%
 \begin{eqnarray}
 \label{eqn_transformed_Z}
 Z
 & = &
  \int D\rho_A D\rho_B D\omega_A D\omega_B Dp
  \ e^{-\beta F}
 \end{eqnarray}
%
%
 \begin{eqnarray}
 \label{eqn_free_energy_1}
 \beta F
 & = &
  \int d{\vec r}
   \left [ \rho_0^{-1}
   (\chi_{AB} \rho_A \rho_B + \chi_{WA} \varrho_W \rho_A + \chi_{WB} \varrho_W \rho_B ) \right . \\
  \nonumber
  & - & i w_A \rho_A - i w_B \rho_B - i p (\rho_0 - \varrho_W - \rho_A - \rho_B) \left . \right ] \\
 \nonumber
  & - & n \ln Q[i w_A,i w_B;N]
 \end{eqnarray}

%
with
 \begin{eqnarray}
 \label{eqn_single_chain_Q}
 Q
 & = &
  \frac{
  \int {\tilde D} {\vec r}_{\alpha} \
  \exp
  \left \lbrace
    - \int_0^f ds \ i N w_A({\vec r}_{\alpha}(s))
    - \int_f^1 ds \ i N w_B({\vec r}_{\alpha}(s))
  \right \rbrace}{\int {\tilde D} {\vec r}_{\alpha}}
 \end{eqnarray}
%
The field $p$ is a ``pressure field'' introduced to enforce
incompressibility and $Q$ is the single-chain partition function
which is a functional of the chemical potential fields $w_A$,
$w_B$ and a function of the chain length $N$.
The explicit ${\vec r}$ dependence is ignored for brevity and ``ideal gas'' terms of
order $\ln V$ have been dropped as they merely add a constant
shift in the free-energy.
The equation for $Q$ is analogous to the
Feynman-Kac formula in the path-integral description of quantum
mechanics \cite{feynman_hibbs65_book} and may be expressed as
%(eqn Q)
%
 \begin{eqnarray}
 \label{eqn_Q}
  Q & = &
   V^{-1} \int d{\vec r} \ q({\vec r},1)
 \end{eqnarray}
%
where $q({\vec r},s)$ is a restricted chain partition function that
may be calculated as the solution to the modified diffusion equation
%(eqn diffusion for q)
%
 \begin{eqnarray}
 \label{eqn_diffusion_for_q}
  \frac{\partial q}{\partial s}
   & = &
   \left \{ \begin{array}{cc}
    R_{{\rm g}0}^2 \nabla^2 q({\vec r},s) - i N w_A q({\vec r},s),
    & 0 < s < f \\
    \\
    R_{{\rm g}0}^2 \nabla^2 q({\vec r},s) - i N w_B q({\vec r},s),
    & f < s < 1
   \end{array} \right .
 \end{eqnarray}
%
subject to the initial condition $q({\vec r},0)$=$1$.
The HS transformation has allowed the discrete density operators ${\hat \rho}$
to be replaced by the ``smeared out'' density fields $\rho$
and permits the chain-chain interactions present in
Eqn.~\ref{eqn_untransformed_Z} to be reformulated in terms of a {\it single} chain
interacting with the chemical potential fields $w_A$ and $w_B$.


However, the functional integrals in Eqn.~\ref{eqn_transformed_Z} are still
analytically intractable, so a numerical simulation is required.
Here, we consider a mean-field approximation where the full
partition function is approximated by its value when the fields
attain their ``saddle-point'' values.
For field-theoretic polymer models such
as the one presented here, it is known that \cite{glenn_review_02}
the equilibrium saddle-points are located along the imaginary axis in
the complex-$w$ plane. 
Hence, the following fields may be rescaled
as $\omega_A$=$i N w_A$, $\omega_B$=$i N w_B$, and $p$  =$i N p$
with $\omega_A$, $\omega_B$ and $p$ purely real.
The simulation results for all density configurations in the phase-separated
block copolymer systems will be presented in terms of the
dimensionless monomer volume fraction $\phi_i$=$\rho_i / \rho_0$ for each species $i$.
{\bf
Typically for bulk systems, the free-energy is rescaled by
$N/\rho_0 V$ and shifted so that $F-F_0$=$0$ for $\omega_i$=$0$
(where $F_0$ is the free-energy of the disordered phase).
For a general $\varrho_W({\vec r})$ it is difficult to analytically determine
what the disordered phase would be and so it is left out of the current
derivation. This only has consequences for measuring the free-energy of
the phase segregated state and therefore the disordered free energy will
be estimated numerically and removed from any diagnostics within the simulation.
}
This gives the free-energy per chain as
%(eqn free energy shifted)
%
 \begin{eqnarray}
 {\tilde F}
 \nonumber
 & = &
  V^{-1} \int d{\vec r}
   \left [
     \chi_{AB} N \phi_A) \phi_B
   + \chi_{WA} N \phi_W \phi_A
   + \chi_{WB} N \phi_W \phi_B \right . \\
  & - &
  \omega_A \phi_A - \omega_B \phi_B - p(1 - \phi_W - \phi_A - \phi_B) \left . \right ]
    - \ln Q[\omega_A,\omega_B]
 \end{eqnarray}
%

%
where $\bar{\phi_A}$ [$\bar{\phi_B}$] is the average
volume fraction of the $A$[$B$] species respectively

The value of
the fields [$\phi_A$,$\phi_B$,$\omega_A$,$\omega_B$,$p$] at the
saddle-point satisfy the following set of equations
%(eqn self consistent set 1-5)
%
 \begin{eqnarray}
   \label{eqn_self_consistent_set_1}
   \omega_A({\vec r}) =
   \chi_{AB} N \ \phi_B({\vec r}) + p({\vec r}) + \chi_{WA} N \phi_W({\vec r})
 \end{eqnarray}
%
 \begin{eqnarray}
   \label{eqn_self_consistent_set_2}
   \omega_B({\vec r}) = 
   \chi_{AB} N \ \phi_A({\vec r}) + p({\vec r}) + \chi_{WB} N \phi_W({\vec r})
 \end{eqnarray}
%
 \begin{eqnarray}
   \label{eqn_self_consistent_set_3}
   \phi_A({\vec r}) + \phi_B({\vec r}) = 1 - \phi_W({\vec r})
 \end{eqnarray}
%
 \begin{eqnarray}
   \label{eqn_self_consistent_set_4}
   \phi_A({\vec r}) = - \frac{V}{Q} \frac{\delta Q}{\delta \omega_A}
 \end{eqnarray}
%
 \begin{eqnarray}
   \label{eqn_self_consistent_set_5}
   \phi_B({\vec r}) = - \frac{V}{Q} \frac{\delta Q}{\delta \omega_B} .
 \end{eqnarray}
%
with
 \begin{eqnarray}
  V = \int d{\vec r} \left [ 1 - \phi_W({\vec r}) \right ]
 \end{eqnarray}

\pagebreak


Using a well-known factorization of the single-chain path integral
\cite{freed72,feynman_hibbs65_book,helfand74} the functional derivatives in
Eqns.~\ref{eqn_self_consistent_set_4} and \ref{eqn_self_consistent_set_5}
may be rewritten, resulting in the following expressions for the $A$ and $B$
monomer density fractions
%(eqn density A)
%(eqn density B)
%
 \begin{eqnarray}
   \label{eqn_density_A}
   \phi_A({\vec r})
   & = &
   \frac{1}{Q} \int_0^f ds \ q({\vec r},s) \ q^\dagger({\vec r},s) \\
   \label{eqn_density_B}
   \phi_B({\vec r})
   & = &
   \frac{1}{Q} \int_f^1 ds \ q({\vec r},s) \ q^\dagger({\vec r},s)
 \end{eqnarray}
%
where the solution to the restricted partition function $q^\dagger$,
may be calculated as the solution to a modified diffusion equation similar
to Eqn.~\ref{eqn_diffusion_for_q} subject to the initial condition
$q^\dagger({\vec r},1)$=$1$ \cite{matsen_schick94}.


%%%%%%%%%%%%%%%%%%%%%%%%%%%%%%%%%%%%%%%%%%%%%%%%%%%%%%%%%
\section{Numerical SCFT Algorithm}
\label{sec_scft_algorithm}
%%%%%%%%%%%%%%%%%%%%%%%%%%%%%%%%%%%%%%%%%%%%%%%%%%%%%%%%%


%%%%%%%%%%%%%%%%%%%%%%%%%%%%%%%%%%%%%%%%%%%%%%%%
\subsection{Calculating saddle-points}
\label{subsec_saddle_points}
%%%%%%%%%%%%%%%%%%%%%%%%%%%%%%%%%%%%%%%%%%%%%%%%


Equations \ref{eqn_self_consistent_set_1} -
\ref{eqn_self_consistent_set_5} must be solved self-consistently
to determine the saddle-point configurations of the fields
[$\phi_A$,$\phi_B$,$\omega_A$,$\omega_B$,$p$].
We want to devise
an iterative procedure whereby the fields may be relaxed towards
their saddle-point values.
Since one is concerned only with the
equilibrium values for the fields, the intermediate field values
are not required to describe any sort of realistic dynamics during
the relaxation algorithm.
Hence, the chemical potential fields
$\omega_A$, $\omega_B$ may be treated as the relevant dynamical
variables with the changes in the $\phi_A$,$\phi_B$ and $p$ fields
treated as fast modes and, at each relaxation step, slaved to the
values of the chemical potentials.
Following the work of Drolet and Fredrickson \cite{drolet99,glenn_review_02} we
employ a ``model A'' type relaxation dynamics
\cite{halperin_rev77} for finding the saddle-point configurations of the fields.
This results in the following expressions for updating the chemical potential
fields from relaxation step $n$ to $n+1$
%
 \begin{eqnarray}
 \label{eq_relaxation_scheme_2}
   \omega_A^{n+1} - \omega_A^{n}
   & = &
    \lambda^{'} \frac{\delta {\tilde F}}{\delta \phi_B^n} +
    \lambda     \frac{\delta {\tilde F}}{\delta \phi_A^n} \\
%   \omega_A^{n+1} - \omega_A^{n}
   & = &
   \nonumber
    \lambda^{'} \left [ \chi_{AB} N \phi_A^n +
                        \chi_{WB} N \phi_W - \omega_B^n + p^n  \right ] \\
   \nonumber
   & + &
    \lambda     \left [ \chi_{AB} N \phi_B^n +
                        \chi_{WA} N \phi_W - \omega_A^n + p^n \right ]
 \end{eqnarray}
%
 \begin{eqnarray}
 \label{eq_relaxation_scheme_3}
   \omega_B^{n+1} - \omega_B^{n}
   & = &
    \lambda         \frac{\delta {\tilde F}}{\delta \phi_B^n} +
    \lambda^{'}     \frac{\delta {\tilde F}}{\delta \phi_A^n} \\
%   \omega_B^{n+1} - \omega_B^{n}
   & = &
   \nonumber
    \lambda     \left [ \chi_{AB} N (\phi_A^n - \bar{\phi}_A) +
                        \chi_{WB} N \phi_W - \omega_B^n + p^n  \right ] \\
   \nonumber
   & + &
    \lambda^{'} \left [ \chi_{AB} N (\phi_B^n - \bar{\phi}_B) +
                        \chi_{WA} N \phi_W - \omega_A^n + p^n \right ]
 \end{eqnarray}
%
where the relaxation parameters are chosen such that
$\lambda^{'} < \lambda$ and $\lambda > 0$
and the quantities $\phi_A^n$, $\phi_B^n$
are calculated as functionals of $\omega_A^n$, $\omega_B^n$
using Eqns.~\ref{eqn_density_A}, \ref{eqn_density_B}.
The relaxation scheme for calculating the saddle-point values for the fields
is implemented through the following steps;
%
 \begin{enumerate}
  \item  random initial values are set for $\omega_A$, $\omega_B$ and $p$
  \item  the modified diffusion equations are solved
         numerically to calculate $q({\vec r},s)$ and $q^\dagger({\vec r},s)$,
  \item  these functions are substituted into Eqns.~\ref{eqn_density_A},
         \ref{eqn_density_B} to obtain $\phi_A^n$, $\phi_B^n$,
  \item  the expressions in Eqns.~\ref{eq_relaxation_scheme_2}, \ref{eq_relaxation_scheme_3}
         are used to update the
         chemical potential fields at the $n$-th iteration, $\omega^{n}$,
         to their values at the $(n+1)$-th iteration, $\omega^{n+1}$,
  \item  the pressure field is updated using the expression
         \begin{eqnarray}
           p^{n+1} = \frac{1}{2} 
                      \left [ \omega^{n+1}_A + \omega^{n+1}_B +
                      (\chi_{AB}-\chi_{WA}-\chi_{WB}) N \phi_W \right ]
	 \end{eqnarray}
 \end{enumerate}
After updating the pressure field, its spatial average
$V^{-1} \int p({\vec r}) \ d{\vec r}$ is subtracted so as
to improve the algorithm's stability.
This has no effect on the equilibrium structure of the chains as the thermodynamic
properties are invariant to a constant shift in the pressure field.
With the new fields $\omega_A$, $\omega_B$ and $p$ the procedure returns
to step $2$ and is repeated until the saddle-point configurations are found.


%%%%%%%%%%%%%%%%%%%%%%%%%%%%%%%%%%%%%%%%%%%%%%%%%%%%%%%%%
%\section{Acknowledgment}
%%%%%%%%%%%%%%%%%%%%%%%%%%%%%%%%%%%%%%%%%%%%%%%%%%%%%%%%%



%%%%%%%%%%%%%%%%%%%%%%%%%%%%% BIBLIOGRAPHY %%%%%%%%%%%%%%%%%%%%%%%%%%%%%
%% bibtex macromolecule
\bibliographystyle{./elsart-num}
\bibliography{confine}
%%%%%%%%%%%%%%%%%%%%%%%%%%%%%%%%%%%%%%%%%%%%%%%%%%%%%%%%%%%%%%%%%%%%%%%%


\end{document}
