%%\documentclass[onecolumn,preprint,showpacs,preprintnumbers,amsmath,amssymb,floatfix]{elsart}
\documentclass[onecolumn,amsmath,amssymb,floatfix]{elsart}

\usepackage{epsfig}


\begin{document}

% Elsevier change
\begin{frontmatter}


\title
{
Parallel algorithm for numerical self-consistent field theory
simulations of block copolymer structure
}

% Elsevier change
% \date{\today}

% Elsevier change, author list

\author{Scott W. Sides\corauthref{cor}},
\ead{swsides@mrl.ucsb.edu}
\author{Glenn H. Fredrickson}

\corauth[cor]{Corresponding author}



% Elsevier change
\address
{
University of California at Santa Barbara (UCSB)
Santa Barbara, CA 93106
}


\begin{abstract}

An efficient algorithm is presented for numerically evaluating a
self-consistent field theoretic (SCFT) model of block copolymer
structure.
This algorithm is implemented on a distributed memory
parallel cluster in order to solve the SCFT equations on large
computational grids.
Simulation results are presented for a
two-component molten mixture of a symmetric $ABA$ triblock
copolymer with an $A$ homopolymer.
These results illustrate
a case in which simulating a large system is required to resolve
features with a wide range of length-scales.

\end{abstract}

\begin{keyword}
block copolymer \sep self-consistent field theory \sep polymer melts \sep parallel computing
\PACS 61.25.Hq \sep 61.43.Bn \sep 64.75.+g \sep 64.60.Cn \sep 64.70.Nd
\end{keyword}

% Elsevier change
% \pacs
% {
% 61.25.Hq, 61.43.Bn, 64.75.+g, 64.60.Cn, 64.70.Nd
% }

% 61.25.Hq  Macromolecular and polymer solutions; polymer melts; swelling
% 61.43.Bn Structural modeling: serial-addition models, computer simulation
% 64.75.+g Solubility, segregation, and mixing; phase separation
% 64.60.Cn Order-isorder transformations; statistical mechanics of model systems
% 64.70.Nd Structural transitions in nanoscale materials

% Elsevier change
% \maketitle


% Elsevier change
\end{frontmatter}


%%%%%%%%%%%%%%%%%%%%%%%%%%%%%%%%%%%%%%%%%%%%%%%%%%%%%%%%%
\section{Introduction}
\label{sec_intro}
%%%%%%%%%%%%%%%%%%%%%%%%%%%%%%%%%%%%%%%%%%%%%%%%%%%%%%%%%


Block copolymers are an important class of materials
that are employed in colloidal stabilization
\cite{napper83_book,russell89_book}, as adhesives, and as
thermoplastic elastomers.
In another application, the adhesion between two glassy,
incompatible homopolymer melts can be strengthened
\cite{brown93_2,kramer94_adhesion_review,kramer96_copoly} by the
addition of diblock copolymers that organize near the interface.
Blends of homopolymers can also be stabilized by the addition of
block copolymers \cite{riess_review75} and can form bi-continuous
microemulsions \cite{bates_microemulsion01}.
The morphologies of pure block copolymer systems observed in experiments
\cite{hamley98_book} and predicted by simulations \cite{drolet99}
suggest many exciting possibilities for creating novel and
potentially useful soft material nanostructures and templates.


Self-consistent field theory (SCFT) for dense polymer melts
\cite{freed72,hong_noolandi81} has been highly successful in
describing complex morphologies in block copolymers.
SCFT provides a method whereby the Hamiltonian of a complex system may be
transformed into a coarse-grained field theory description, whose
mean-field solution is amenable to a battery of analytic and
numerical methods.
Matsen and Schick \cite{matsen_schick94} have
solved the SCFT equations with a ``spectral'' method that can
calculate very accurate free-energies, but requires foreknowledge
of the preferred symmetry for each structural phase.
Recently, a ``real space'' numerical method has been used to solve the SCFT
equations \cite{drolet99,fraaije97,glenn_review_02} which does not need any
information about the equilibrium morphologies.
This numerical SCFT method is more flexible than the spectral method and has a
greater predictive capability, however it is computationally
intensive and requires large memory resources.


For a single-component, monodisperse copolymer, the ordered
structures are periodic and have a size on the order of the radius
of gyration of the unperturbed chain.
These cases require simulating relatively small systems in order to resolve the
relevant length scale.
However for multicomponent mixtures or
systems with chemical disorder, a range of widely separated length
scales could be germane to fully describing equilibrium
morphologies and will require the use of large simulation cells.
The present work is aimed at
providing a computational methodology appropriate for such systems.
Section \ref{sec_theory} contains an outline of  SCFT for
polymers.
Section \ref{sec_scft_algorithm} briefly describes the
numerical implementation of the theory including details on the
parallelization algorithm.
Results for a triblock/homopolymer mixture are discussed in
Sec.~\ref{sec_results} followed by conclusions in Sec.~\ref{sec_conclusions}.



%%%%%%%%%%%%%%%%%%%%%%%%%%%%%%%%%%%%%%%%%%%%%%%%%%%%%%%%%
\section{SCFT theory for polymers}
\label{sec_theory}
%%%%%%%%%%%%%%%%%%%%%%%%%%%%%%%%%%%%%%%%%%%%%%%%%%%%%%%%%


We briefly describe the model and simulation method used to predict
the structure of a dense melt of $AB$ diblock copolymers.
The method is easily generalized to
treat the $ABA$ triblock / $A$ homopolymer mixtures considered in
this paper.
The Hamiltonian for a dense system of polymer chains
comprised of blocks containing chemically distinct monomers $A$ and $B$ is
%
 \begin{eqnarray}
 % Elsevier change XXX ALL \cal GONE
 H
 & = &
  \frac{1}{4 R_{{\rm g}0}^2 } \sum^{n}_{\alpha = 1}
           \int_0^1 ds \left ( \frac{ d {\vec r}_{\alpha}(s)}{ds} \right )^{2} +
           \rho_0^{-1} \int d{\vec r} \ \chi \ {\hat \rho_A}({\vec r}) {\hat \rho_B}({\vec r}) .
 \end{eqnarray}
%
The first term in $H$ is the free-energy contribution from
the so-called ``Gaussian thread'' \cite{doi_book} model.
This term can be
derived as the continuous limit of a chain model consisting of
coarse-grained, spherical beads connected by harmonic springs,
where $R_{{\rm g}0}$ is the radius of gyration for the unperturbed
Gaussian chain, $n$ is the number of chains and ${\vec
r}_{\alpha}(s)$ is a space curve parameterized by the chain
contour variable $s$ which describes the configuration of the
$\alpha^{th}$ chain. The second term is obtained from a Flory-type
model of the interaction energy between chemically distinct
monomer species $A$ and $B$, which is parameterized by $\chi$. The
total monomer density is $\rho_0$=$n N/V$, where $N$ is the
polymerization index and $V$ is the volume. For these simulations
$\chi > 0$, indicating an overall repulsive force between
dissimilar monomers which drives phase separation. Note that while
$\chi$ is known experimentally to depend on temperature, there is
no such explicit dependence in the present theoretical
formulation. The predictions for the mesoscopic structure of the
triblock system studied in this paper are for particular values of
$\chi$, which depend not only on the specific interactions between
chemical species $A$ and $B$, but on the temperature as well.


The monomer density operators for the $A$ and $B$ species are
%
 \begin{eqnarray}
 {\hat \rho_A}({\vec r})
     & = & N \sum^{n}_{\alpha = 1}
           \int_0^f ds \ \delta({\vec r} - {\vec r}_{\alpha}(s)) \\
 {\hat \rho_B}({\vec r})
     & = & N \sum^{n}_{\alpha = 1}
           \int_f^1 ds \ \delta({\vec r} - {\vec r}_{\alpha}(s))
 \end{eqnarray}
%
where $f$ is the fraction of the chain consisting of $A$ monomers.
The partition function for this system can be written as
%(eqn untransformed Z)
%
 \begin{eqnarray}
 \label{eqn_untransformed_Z}
  Z & = &
      \int \prod^{n}_{\alpha = 1} \tilde{D} {\vec r}_{\alpha} \
      \exp \left \lbrace
           - \rho_0^{-1} \int d{\vec r} \ \chi \ {\hat \rho_A}({\vec r}) {\hat \rho_B}({\vec r})
           \right \rbrace
      \delta[\rho_0 - {\hat \rho_A} - {\hat \rho_B}]
 \end{eqnarray}
%
with
%
 \begin{eqnarray}
  \tilde{D} {\vec r}_{\alpha}
    & = &
    D {\vec r}_{\alpha}
      \exp \left \lbrace
           - \frac{1}{4 R_{{\rm g}0}^2 } \int_0^1
           ds \left ( \frac{ d {\vec r}_{\alpha}(s)}{ds} \right )^{2}
           \right \rbrace
 \end{eqnarray}
where $\int D {\vec r}_\alpha$ denotes a path integral over
all possible conformations of the $\alpha$th chain and the delta
function enforces an incompressibility constraint between $A$
and $B$ monomers such that the total monomer density is kept
constant.
The partition function is now expressed in terms of
monomer density operators which enumerate the positions of the
$A$/$B$ monomers on the $n$ polymer chains.
This expression for
$Z$ may be converted into a field-theoretic description by a
Hubbard-Stratonovich (HS) transformation
by insertion of the functional integral identity
%(eqn functional identity)
%
 \begin{eqnarray}
 \label{eqn_functional_identity}
  1
  & = &
  \int D[\rho] \ \delta[\rho - {\hat \rho} ] \\
  \nonumber
  & = &
  \int D[\rho] D[\omega]
    \exp
    \left \lbrace
     \int d{\vec r} \ i w({\vec r})
       \left [ \rho({\vec r}) - {\hat \rho}({\vec r}) \right ]
     \right \rbrace
 \end{eqnarray}
%
where the chemical potential field $w({\vec r})$ has been introduced through
the exponential representation of the delta functional
$\delta[\rho - {\hat \rho} ]$.
Using the identity in Eqn.~\ref{eqn_functional_identity} for both $A$ and $B$ density
operators and substituting the exponential form of the incompressibility constraint
gives
%(eqn transformed Z)
%(eqn free energy 1)
%(eqn single chain Q)
%
 \begin{eqnarray}
 \label{eqn_transformed_Z}
 Z
 & = &
  \int D\rho_A D\rho_B D\omega_A D\omega_B Dp
  \ e^{-\beta F}
 \end{eqnarray}
%
%
 \begin{eqnarray}
 \label{eqn_free_energy_1}
 \beta F
 & = &
  \int d{\vec r}
   \left [
     \rho_0^{-1} \chi \rho_A \rho_B
      - i w_A \rho_A
      - i w_B \rho_B - i p (\rho_0 - \rho_A - \rho_B) \right ] \\
 & - & n \ln Q[i w_A,i w_B;N]
 \end{eqnarray}
%
with
 \begin{eqnarray}
 \label{eqn_single_chain_Q}
 Q
 & = &
  \frac{
  \int {\tilde D} {\vec r}_{\alpha} \
  \exp
  \left \lbrace
    - \int_0^f ds \ i N w_A({\vec r}_{\alpha}(s))
    - \int_f^1 ds \ i N w_B({\vec r}_{\alpha}(s))
  \right \rbrace}{\int {\tilde D} {\vec r}_{\alpha}}
 \end{eqnarray}
%
The field $p$ is a ``pressure field'' introduced to enforce
incompressibility and $Q$ is the single-chain partition function
which is a functional of the chemical potential fields $w_A$,
$w_B$ and a function of the chain length $N$.
The explicit ${\vec r}$ dependence is ignored for brevity and ``ideal gas'' terms of
order $\ln V$ have been dropped as they merely add a constant
shift in the free-energy.
The equation for $Q$ is analogous to the
Feynman-Kac formula in the path-integral description of quantum
mechanics \cite{feynman_hibbs65_book} and may be expressed as
%(eqn Q)
%
 \begin{eqnarray}
 \label{eqn_Q}
  Q & = &
   V^{-1} \int d{\vec r} \ q({\vec r},1)
 \end{eqnarray}
%
where $q({\vec r},s)$ is a restricted chain partition function that
may be calculated as the solution to the modified diffusion equation
%(eqn diffusion for q)
%
 \begin{eqnarray}
 \label{eqn_diffusion_for_q}
  \frac{\partial q}{\partial s}
   & = &
   \left \{ \begin{array}{cc}
    R_{{\rm g}0}^2 \nabla^2 q({\vec r},s) - i N w_A q({\vec r},s),
    & 0 < s < f \\
    \\
    R_{{\rm g}0}^2 \nabla^2 q({\vec r},s) - i N w_B q({\vec r},s),
    & f < s < 1
   \end{array} \right .
 \end{eqnarray}
%
subject to the initial condition $q({\vec r},0)$=$1$.
The HS transformation has allowed the discrete density operators ${\hat \rho}$
to be replaced by the ``smeared out'' density fields $\rho$
and permits the chain-chain interactions present in
Eqn.~\ref{eqn_untransformed_Z} to be reformulated in terms of a {\it single} chain
interacting with the chemical potential fields $w_A$ and $w_B$.


However, the functional integrals in Eqn.~\ref{eqn_transformed_Z} are still
analytically intractable, so a numerical simulation is required.
Here, we consider a mean-field approximation where the full
partition function is approximated by its value when the fields
attain their ``saddle-point'' values.
For field-theoretic polymer models such
as the one presented here, it is known that \cite{glenn_review_02}
the equilibrium saddle-points are located along the imaginary axis in
the complex-$w$ plane. 
Hence, the following fields may be rescaled
as $\omega_A$=$i N w_A$, $\omega_B$=$i N w_B$, and $p$  =$i N p$
with $\omega_A$, $\omega_B$ and $p$ purely real.
The simulation results for all density configurations in the phase-separated
block copolymer systems will be presented in terms of the
dimensionless monomer volume fraction $\phi_i$=$\rho_i / \rho_0$ for each species $i$.
Rescaling the free-energy by
$N/\rho_0 V$ and shifting so that $F-F_0$=$0$ for $\omega_i$=$0$
(where $F_0$ is the free-energy of the disordered phase),
gives the free-energy per chain as
%(eqn free energy shifted)
%
 \begin{eqnarray}
 \label{eqn_free_energy_shifted}
 {\tilde F}
 \nonumber
 & = &
  V^{-1} \int d{\vec r}
   \left [
     \chi N (\phi_A - \bar{\phi}_A)
            (\phi_B - \bar{\phi}_B)
     - \omega_A \phi_A
     - \omega_B \phi_B
     - p(1 - \phi_A - \phi_B)
   \right ] \\
 & - & \ln Q[\omega_A,\omega_B]
 \end{eqnarray}
%
where $\bar{\phi_A}$ [$\bar{\phi_B}$] is the average
volume fraction of the $A$[$B$] species respectively.
The value of
the fields [$\phi_A$,$\phi_B$,$\omega_A$,$\omega_B$,$p$] at the
saddle-point satisfy the following set of equations
%(eqn self consistent set 1-5)
%
 \begin{eqnarray}
   \label{eqn_self_consistent_set_1}
   \omega_A({\vec r}) = \chi N \ (\phi_B({\vec r})-\bar{\phi}_B) + p({\vec r})
 \end{eqnarray}
%
 \begin{eqnarray}
   \label{eqn_self_consistent_set_2}
   \omega_B({\vec r}) = \chi N \ (\phi_A({\vec r})-\bar{\phi}_A) + p({\vec r})
 \end{eqnarray}
%
 \begin{eqnarray}
   \label{eqn_self_consistent_set_3}
   \phi_A({\vec r}) + \phi_B({\vec r}) = 1
 \end{eqnarray}
%
 \begin{eqnarray}
   \label{eqn_self_consistent_set_4}
   \phi_A({\vec r}) = - \frac{V}{Q} \frac{\delta Q}{\delta \omega_A}
 \end{eqnarray}
%
 \begin{eqnarray}
   \label{eqn_self_consistent_set_5}
   \phi_B({\vec r}) = - \frac{V}{Q} \frac{\delta Q}{\delta \omega_B} .
 \end{eqnarray}
%
Using a well-known factorization of the single-chain path integral
\cite{freed72,feynman_hibbs65_book,helfand74} the functional derivatives in
Eqns.~\ref{eqn_self_consistent_set_4} and \ref{eqn_self_consistent_set_5}
may be rewritten, resulting in the following expressions for the $A$ and $B$
monomer density fractions
%(eqn density A)
%(eqn density B)
%
 \begin{eqnarray}
   \label{eqn_density_A}
   \phi_A({\vec r})
   & = &
   \frac{1}{Q} \int_0^f ds \ q({\vec r},s) \ q^\dagger({\vec r},s) \\
   \label{eqn_density_B}
   \phi_B({\vec r})
   & = &
   \frac{1}{Q} \int_f^1 ds \ q({\vec r},s) \ q^\dagger({\vec r},s)
 \end{eqnarray}
%
where the solution to the restricted partition function $q^\dagger$,
may be calculated as the solution to a modified diffusion equation similar
to Eqn.~\ref{eqn_diffusion_for_q} subject to the initial condition
$q^\dagger({\vec r},1)$=$1$ \cite{matsen_schick94}.


%%%%%%%%%%%%%%%%%%%%%%%%%%%%%%%%%%%%%%%%%%%%%%%%%%%%%%%%%
\section{Numerical SCFT Algorithm}
\label{sec_scft_algorithm}
%%%%%%%%%%%%%%%%%%%%%%%%%%%%%%%%%%%%%%%%%%%%%%%%%%%%%%%%%


%%%%%%%%%%%%%%%%%%%%%%%%%%%%%%%%%%%%%%%%%%%%%%%%
\subsection{Calculating saddle-points}
\label{subsec_saddle_points}
%%%%%%%%%%%%%%%%%%%%%%%%%%%%%%%%%%%%%%%%%%%%%%%%


Equations \ref{eqn_self_consistent_set_1} -
\ref{eqn_self_consistent_set_5} must be solved self-consistently
to determine the saddle-point configurations of the fields
[$\phi_A$,$\phi_B$,$\omega_A$,$\omega_B$,$p$].
We want to devise
an iterative procedure whereby the fields may be relaxed towards
their saddle-point values.
Since one is concerned only with the
equilibrium values for the fields, the intermediate field values
are not required to describe any sort of realistic dynamics during
the relaxation algorithm.
Hence, the chemical potential fields
$\omega_A$, $\omega_B$ may be treated as the relevant dynamical
variables with the changes in the $\phi_A$,$\phi_B$ and $p$ fields
treated as fast modes and, at each relaxation step, slaved to the
values of the chemical potentials.
Following the work of Drolet and Fredrickson \cite{drolet99,glenn_review_02} we
employ a ``model A'' type relaxation dynamics
\cite{halperin_rev77} for finding the saddle-point configurations of the fields.
This results in the following expressions for updating the chemical potential
fields from relaxation step $n$ to $n+1$
%
 \begin{eqnarray}
 \label{eq_relaxation_scheme_2}
   \omega_A^{n+1} - \omega_A^{n}
   & = &
    \lambda^{'} \frac{\delta {\tilde F}}{\delta \phi_B^n} +
    \lambda     \frac{\delta {\tilde F}}{\delta \phi_A^n} \\
   \omega_A^{n+1} - \omega_A^{n}
   & = &
   \nonumber
    \lambda^{'} \left [ \phi_A^n - \bar{\phi}_A
                      - \frac{\omega_B^n - p^n}{\chi N}
                \right ] +
    \lambda     \left [ \phi_B^n - \bar{\phi}_B
                      - \frac{\omega_A^n - p^n}{\chi N}
                \right ]
 \end{eqnarray}
%
 \begin{eqnarray}
 \label{eq_relaxation_scheme_3}
   \omega_B^{n+1} - \omega_B^{n}
   & = &
    \lambda         \frac{\delta {\tilde F}}{\delta \phi_B^n} +
    \lambda^{'}     \frac{\delta {\tilde F}}{\delta \phi_A^n} \\
   \omega_B^{n+1} - \omega_B^{n}
   & = &
   \nonumber
    \lambda     \left [ \phi_A^n - \bar{\phi}_A
                      - \frac{\omega_B^n - p^n}{\chi N}
                \right ] +
    \lambda{'}  \left [ \phi_B^n - \bar{\phi}_B
                      - \frac{\omega_A^n - p^n}{\chi N}
                \right ]
 \end{eqnarray}
%
where the relaxation parameters are chosen such that
$\lambda^{'} < \lambda$ and $\lambda > 0$
and the quantities $\phi_A^n$, $\phi_B^n$
are calculated as functionals of $\omega_A^n$, $\omega_B^n$
using Eqns.~\ref{eqn_density_A}, \ref{eqn_density_B}.
The relaxation scheme for calculating the saddle-point values for the fields
is implemented through the following steps;
%
 \begin{enumerate}
  \item  random initial values are set for $\omega_A$, $\omega_B$ and $p$
  \item  the modified diffusion equations are solved
         numerically to calculate $q({\vec r},s)$ and $q^\dagger({\vec r},s)$,
  \item  these functions are substituted into Eqns.~\ref{eqn_density_A},
         \ref{eqn_density_B} to obtain $\phi_A^n$, $\phi_B^n$,
  \item  the expressions in Eqns.~\ref{eq_relaxation_scheme_2}, \ref{eq_relaxation_scheme_3}
         are used to update the
         chemical potential fields at the $n$-th iteration, $\omega^{n}$,
         to their values at the $(n+1)$-th iteration, $\omega^{n+1}$,
  \item  the pressure field is updated using the expression
         $p^{n+1}$=$(\omega^{n+1}_A + \omega^{n+1}_B)/2$.
 \end{enumerate}
After updating the pressure field, its spatial average
$V^{-1} \int p({\vec r}) \ d{\vec r}$ is subtracted so as
to improve the algorithm's stability.
This has no effect on the equilibrium structure of the chains as the thermodynamic
properties are invariant to a constant shift in the pressure field.
With the new fields $\omega_A$, $\omega_B$ and $p$ the procedure returns
to step $2$ and is repeated until the saddle-point configurations are found.




%%%%%%%%%%%%%%%%%%%%%%%%%%%%%%%%%%%%%%%%%%%%%%%%
\subsection{Parallel algorithm}
\label{subsec_parallel}
%%%%%%%%%%%%%%%%%%%%%%%%%%%%%%%%%%%%%%%%%%%%%%%%


The most computationally demanding step in the SCFT algorithm
outlined in Sec.~\ref{subsec_saddle_points} is in obtaining the
solutions for the restricted partition functions
$q$ and $q^\dag$.
Finding these solutions requires solving
Eqn.~\ref{eqn_diffusion_for_q} every time the chemical potential
fields $\omega_A$ and $\omega_B$ are updated.
To relax these
fields to the saddle-point configurations can require on the order
of $10^3$-$10^4$ iterations; hence, an efficient method for
solving the modified diffusion equations for $q({\vec r},s)$ and
$q^{\dagger}({\vec r},s)$ is essential.
We have tried various
finite-difference algorithms including explicit and implicit
schemes \cite{caj_fletcher_book} for solving partial differential
equations.
Implicit schemes such as Crank-Nicholson are more
stable and accurate than the simple forward-time centered-space
explicit scheme.
However, the Crank-Nicholson scheme is only
conditionally stable in three dimensions and is not
straightforward to implement on a distributed memory computer
cluster.
More sophisticated explicit schemes such as the
Dufort-Frankel algorithm are unconditionally stable in two and
three dimensions and are more readily amenable to parallelization.
Still, this scheme has accuracy restrictions which limit the size
of the contour step $\Delta s$ that can be used.


In a recent paper \cite{lookman02}, a pseudo-spectral (PS) algorithm was used to
solve the modified diffusion equations in SCFT theory for calculating
block-copolymer structure.
This method is unconditionally stable in all dimensions and has superior accuracy
to the finite-difference algorithms referenced above.
The derivation of the PS algorithm starts with the
exact expression for propagating the solution at $q({\vec r},s)$ to
$q({\vec r},s+\Delta s)$
%(eqn pseudospec 1)
%
 \begin{eqnarray*}
 \label{eqn_pseudospec_1}
  q({\vec r},s + \Delta s)
  & = &
   \exp \left[\Delta s (\nabla^2 - w({\vec r})) \right ]
   q({\vec r},s) .
 \end{eqnarray*}
%
If the time propagator above is expanded and terms of order
$(\Delta s)^3$ are ignored, the remaining terms can be rewritten as
%(eqn pseudospec 2) (eqn pseudospec 3)
%
 \begin{eqnarray*}
 \label{eqn_pseudospec_2}
  q({\vec r},s + \Delta s)
  & \approx &
   e^{- w({\vec r}) \Delta s / 2} e^{\Delta s \nabla^2}
   e^{- w({\vec r}) \Delta s / 2}
   q({\vec r},s) \\
  & \approx &
  \label{eqn_pseudospec_3}
   e^{- w({\vec r}) \Delta s / 2} {\hat F}^{-1}
   \left [ e^{- \Delta s \ k^2}
   {\hat F}
    \left  [ e^{- w({\vec r}) \Delta s / 2} q({\vec r},s)
    \right ]
   \right ] .
 \end{eqnarray*}
%
The first exponential factor depending on $\omega({\vec r})/2$ is applied
in real space to $q({\vec r},s)$.
The Fourier transform ${\hat F}$ is applied to this result and then
the operator $\exp[\Delta s \nabla^2]$ can be applied exactly in reciprocal
space as $\exp(-\Delta s \ k^2)$
for a given discretization of the chemical potential fields $\omega$ and
restricted partition functions $q({\vec r},s)$.
The inverse Fourier transform ${\hat F}^{-1}$ is applied to this result
and the last exponential factor is again applied in real space.
Since two Fourier transforms must be applied to calculate $q({\vec r},s)$ for each
contour step $s$, an efficient Fourier transform algorithm is essential.
The publicly available FFTW \cite{fftw98} fast-Fourier transform libraries
were employed in the numerical SCFT code used to obtain the
results in this paper.


The unconditional stability and improved accuracy
of the PS algorithm allows for fewer spectral elements and
a larger $\Delta s$ to be used, relative to finite-difference methods.
Therefore, fewer contour steps are needed to solve for
$q({\vec r},s)$ from $s$=$0$ to $s$=$1$ resulting in decreased
computation times.
For very large systems, the SCFT algorithm has been parallelized enabling
implementation on a distributed memory cluster of computers.
Figure \ref{fig_domain_decomposition} shows a diagram of a
computational grid used to discretize the fields $\omega_A({\vec r})$,
$\omega_B({\vec r})$, $\phi_A({\vec r})$, $\phi_B({\vec r})$
as well as the functions $q({\vec r},s)$, $q^{\dagger}({\vec r},s)$.
This diagram illustrates a spatial, domain decomposition
scheme in which separate regions of the computational lattice are
assigned to distinct processors.
All variables, including those regions of the computational grid shown in
Fig.~\ref{fig_domain_decomposition}, that have unique values
corresponding to a distinct processor (or node) will be referred
to as ``local''.
Variables that have the same values on all processors or refer to
overall system parameters will be referred to as ``global''.
Each processor performs computations on its local data, and uses MPI
communication between nodes when a
computation requires data which is local to another processor.
The steps in the SCFT algorithm outlined in
Sec.~\ref{subsec_saddle_points} are parallelized by identifying
the steps requiring only local computation, and the steps which
also include inter-processor communication via MPI.


After initializing all fields, step $2$ in Sec.~\ref{subsec_saddle_points}
requires solving modified diffusion equations, whose solutions can be obtained
by repeated evaluation of Eqn.~\ref{eqn_pseudospec_3}.
The action of the real-space operators $\exp(- w({\vec r}) \ \Delta s / 2)$
and the reciprocal-space operator $\exp(- \Delta s \ k^2)$ need only local computation.
However, the forward and backward Fourier transforms require both local computation and
inter-processor communication.
A parallel Fourier transform is performed in an analogous manner as a serial
Fourier transform on a multi-dimensional array of data.
First, a series of  1-d FFT's in the $y$ and $z$ directions is applied to
all local data.
Since FFT algorithms (such as the standard Danielson-Lanczos algorithm \cite{pres92})
require all data in one dimension to be local, this explains the slab decomposition
illustrated in Fig.~\ref{fig_domain_decomposition}.
Second, a
matrix transpose operation is performed and then another 1d FFT is
applied along the remaining $x$ direction.
Communication cost is incurred through the matrix transpose,
which clearly requires sending and receiving data to and from
sections of the computation grid local to different processors.
In step $3$, Eqns.~\ref{eqn_density_A} and \ref{eqn_density_B} are used to
compute the monomer density fields $\phi_A$, $\phi_B$.
The integrals over $ds$ are calculated by a simple quadrature algorithm,
(using the extended-trapezoidal rule) requiring only local computation.
However, the single-chain partition function $Q$ contains an integral
over space requiring more inter-processor communication, which
is implemented with a simple MPI ``all-to-all'' \cite{mpi_book}
global communication call.
The final steps $4$ and $5$,
updating the fields $\omega_A$, $\omega_B$ and $p$ from $n$ to
$n+1$, require only local computation.
Timing results for the
parallel SCFT code are shown in Figs.~\ref{fig_timing}a and
\ref{fig_timing}b for $2d$ and $3d$ systems, respectively.
The data in these figures represents both the
computation/communication time required to solve the diffusion equation
with the pseudo-spectral algorithm for $1000$ field iterations.
Note, that except for I/O, other parts of the SCFT calculation are small
in comparison to this contribution to the CPU time.
The insets in Fig.~\ref{fig_timing} show log-log plots of the same timing data with
dotted lines indicating the limit of perfect speedup.
A decrease in parallel efficiency for small systems is most
clearly illustrated in the inset for Fig.~\ref{fig_timing}b.
For the smaller system ($32^3$), parallel performance is degraded
as the number of processors increases.
This trend
illustrates the increasing ratio of communication-to-computation CPU time
as the number of lattice points per processor decreases.
Hence, the decrease in parallel efficiency for the larger system in
Fig.~\ref{fig_timing}b is less pronounced than in the smaller system.



%%%%%%%%%%%%%%%%%%%%%%%%%%%%%%%%%%%%%%%%%%%%%%%%%%%%%%%%%
\section{Results}
\label{sec_results}
%%%%%%%%%%%%%%%%%%%%%%%%%%%%%%%%%%%%%%%%%%%%%%%%%%%%%%%%%


Figures \ref{fig_density_aba_a_chi_0.10} and
\ref{fig_density_aba_a_chi_0.16} show density configurations for a
mixture of symmetric triblock copolymer chains and monodisperse
homopolymer chains.
The outer blocks of the triblock consist of
equally sized lengths of polymerized $A$ monomers while the inner
block consists of $B$ monomers.
The total number of $A$ monomers
on each triblock chain is $f_A N_t$ where $f_A$ is the fraction of
$A$ monomers/triblock and $N_t$ is the triblock polymerization
index.
The volume fraction of monomers on homopolymer chains $v_h$ is given by
%
 \begin{equation}
 \label{eqn_vol_frac_homopolymer}
   v_h = \frac{n_h N_h}{n_h N_h + n_t N_t}
 \end{equation}
%
where $n_h$($n_t$) is the number of homopolymer(triblock) chains
and $N_h$ is the homopolymer polymerization index.
The chain lengths used are $N_h$=$100$ and $N_t$=$200$.
The total density of $A$ monomers is shown in Figs.~\ref{fig_density_aba_a_chi_0.10} and
\ref{fig_density_aba_a_chi_0.16}
with light(dark) regions corresponding to high(low) values of the
monomer density volume fraction, $\rho_A({\vec r})$.
With only two chemical species, the incompatibility between them
is described by a single Flory parameter, $\chi_{AB}$.


Small values of $v_h$ (on the left of each plot) correspond to a
mixture consisting of almost pure triblock while the large values
of $v_h$ correspond to a nearly pure homopolymer mixture.
For small $v_h$, the typical phase-separated cylindrical and lamellar
ordered structures can be seen with disordered phases for the
largest and smallest values of $f_A$.
As homopolymer is added to
the pure triblock, the $A$ domains of the triblock begin to swell.
For sufficiently large amounts of homopolymer, the $A$ domains of
the triblock can stretch no further and macrophase separation
occurs with large domains of pure homopolymer appearing.
All of these features are in qualitative agreement with SCFT results by
Matsen \cite{matsen_mixtures95} of an $AB$ diblock / $A$
homopolymer mixture.
However, these calculations use a spectral
method to calculate the SCFT theory and require assumptions about
the symmetries of the phase-separated states.
The attractiveness of the present pseudo-spectral approach is that
it allows for a rapid assessment of phase behavior in complex
mixtures characterized by a large number of model parameters.


The density plot shown for $f_A$=$0.3$ and $v_h \sim 0.2$ in
Fig.~\ref{fig_density_aba_a_chi_0.16} illustrates the two-phase region particularly well.
The smallest domains consist primarily of $A$ monomers belonging to
triblock chains with a size corresponding to the cylindrical phase
in the pure triblock.
The larger domains consist mostly of homopolymer chains that have been expelled
from the smaller cylindrical phase domains.
Additional features have been introduced with length scales larger than
the cylindrical triblock domains, such as the spacing between the larger
domains and the arrangement of the small domains which
is clearly affected by the presence of the phase-separated homopolymer-rich regions.


It should be noted that these density profiles are not the true
saddle-point configurations of the fields, but represent local
metastable configurations in the vicinity of the equilibrium
saddle point.
These configurations contain a variety of
topological defects which distort pure lamellar order
(e.g.~$f_A$=$0.50$, $v_h \sim 0.05$) as well as perfect, hexagonal
ordering for cylindrical domains (e.g.~ $f_A$=$0.70$, $v_h \sim 0.05$).
However, ``near-equilibrium'' configurations such as these
are typically observed in experiments.
Moreover, the present simulation strategy has produced defect structures in
qualitative agreement with size adjustments observed in $5$-fold
and $7$-fold coordinated defects in PCHE-PE-PCHE triblock thin films \cite{hammond03}.
Figure \ref{fig_big} shows the SCFT result
for a much larger system ($1024 \times 1024$ lattice) with the same
parameters as one of the runs shown in
Fig.~\ref{fig_density_aba_a_chi_0.16} (i.e.~$f_A$=$0.30$, $v_h \sim 0.20$).
The true equilibrium configuration for a macrophase
separated system would presumably contain a single,
homopolymer-rich domain but Fig.~\ref{fig_big} shows one possible
distribution of sizes and arrangements of smaller,
homopolymer-rich domains.
The distribution of these smaller
domains can be thought of as topological defects in the copolymer
mixture that will migrate and very slowly coalesce into a single
macroscopic domain as additional iterations of the SCFT algorithm are performed.
However the configuration shown in Fig.~\ref{fig_big}
should correspond to an experimentally realizable structure, as in
the triblock thin-film study cited above.



%%%%%%%%%%%%%%%%%%%%%%%%%%%%%%%%%%%%%%%%%%%%%%%%%%%%%%%%%
\section{Conclusions}
\label{sec_conclusions}
%%%%%%%%%%%%%%%%%%%%%%%%%%%%%%%%%%%%%%%%%%%%%%%%%%%%%%%%%


Numerical SCFT has been used to study the complex morphologies
present in phase-separated block copolymer systems. We presented
an efficient algorithm using a combination of a pseudo-spectral
algorithm for solving a modified diffusion equation and
parallelization that enables faster and larger simulations. This
capability allows investigation of systems where many length
scales are needed to fully describe the morphology. A mixture of
$ABA$ triblock and $A$ homopolymer is used to benchmark the SCFT
algorithm and demonstrate the ability to simultaneously resolve
both microphase and macrophase separation. The micellar regimes in
the $ABA$/$A$ mixture could be further studied with a complex
Langevin scheme \cite{glenn_review_02} for sampling full,
thermodynamic averages within SCFT.
If one extends this type of
study to $3d$ systems (in order to resolve the cubic, gyroid and
bcc spherical phases), the parallel algorithm presented here will
be even more helpful.




%%%%%%%%%%%%%%%%%%%%%%%%%%%%%%%%%%%%%%%%%%%%%%%%%%%%%%%%%
\section{Acknowledgment}
%%%%%%%%%%%%%%%%%%%%%%%%%%%%%%%%%%%%%%%%%%%%%%%%%%%%%%%%%

S. W. Sides is grateful to  Atofina Chemicals for financial
support. The authors also acknowledge support from the National
Science Foundation under Award Number DMR-98-70785 and extensive
use of the UCSB-MRL Central Computing Facilities.
Special thanks to Jeffery Barteet for his important role in
building the MRL parallel computing cluster.



%%%%%%%%%%%%%%%%%%%%%%%%%%%%% BIBLIOGRAPHY %%%%%%%%%%%%%%%%%%%%%%%%%%%%%
%% bibtex macromolecule
\bibliographystyle{./elsart-num}
\bibliography{scft}
%%%%%%%%%%%%%%%%%%%%%%%%%%%%%%%%%%%%%%%%%%%%%%%%%%%%%%%%%%%%%%%%%%%%%%%%



%%%%%%%%%%%%%%%%%%%%%%%%%%%  FIGURE 1 %%%%%%%%%%%%%%%%%%%%%%%%%%%%%%%%%%
\begin{figure*}
\centerline
{
\epsfig{figure=parallel_1.eps,angle=0,width=6in}
}
\caption
{
\label{fig_domain_decomposition}
%(fig domain decomposition)
Spatial domain decomposition scheme for solving the
SCFT equations on a distributed-memory computer cluster.
This schematic diagram shows the computational grid used to discretize
the fields $\omega_A({\vec r})$, $\omega_B({\vec r})$,
$\phi_A({\vec r})$, $\phi_B({\vec r})$
as well as the functions $q({\vec r},s)$, $q^{\dagger}({\vec r},s)$.
Each processor is assigned a ``slab'' portion of the grid, of size
$L_x/(N_p)$ $\times$ $L_y$ $\times$ $L_z$ where $N_p$ is the
number of processors.
The size of the grid spacing ($dx$=$dy$=$dz$) is in units of $R_g$,
the radius of gyration for the unperturbed Gaussian chain.
This domain decomposition scheme is easily adapted to $2d$ and $3d$,
any number of processors and larger systems.
}
\vspace{0.26cm}
\end{figure*}
%%%%%%%%%%%%%%%%%%%%%%%%%%%%%%%%%%%%%%%%%%%%%%%%%%%%%%%%%%%%%%%%%%%%%%%%%%



%%%%%%%%%%%%%%%%%%%%%%%%% FIGURE 2 %%%%%%%%%%%%%%%%%%%%%%%%%%%%%
\begin{figure*}
\centerline
{
\epsfig{figure=timing_2d.epsi,angle=270,width=6in}
}
\vspace{1.0cm}
\centerline
{
\epsfig{figure=timing_3d.epsi,angle=270,width=6in}
}
\caption
{
\label{fig_timing}
%(fig timing)
CPU time $t_n$ vs.~number of processors $n$ for
(a) 2d systems and
(b) 3d systems.
The time shown is the total computation and communication time
required to solve the diffusion equation using the pseudo-spectral
algorithm.
This part of the SCFT calculation accounts for the majority of
the total CPU time.
The insets show the same timing data on a log-log plot.
The dotted lines indicate the limit of perfect speedup and have
slopes of $-1$.
Note: for the largest systems, no timing data is available
for one and two processors because the job exceeded local memory.
}
\vspace{0.26cm}
\end{figure*}
%%%%%%%%%%%%%%%%%%%%%%%%%%%%%%%%%%%%%%%%%%%%%%%%%%%%%%%%%%%%%%%%



%%%%%%%%%%%%%%%%%%%%%%%%% FIGURE %%%%%%%%%%%%%%%%%%%%%%%%%%%%%%%
\begin{figure*}
\centerline
{
\epsfig{figure=chi_0.10_beta_0.5.epsi,angle=270,width=6in}
}
\caption
{
\label{fig_density_aba_a_chi_0.10}
%(fig density aba a chi 0.10)
Density configurations of symmetric ABA triblock / A homopolymer mixtures
for $\chi_{AB}$=$0.10$.
White regions denote a high concentration of $A$ monomers while
dark regions denote a high concentration of $B$ monomers.
The chain polymerization indexes are $N_t$=$200$ and $N_h$=$100$.
The vertical direction in each figure shows varying amounts of $A$ monomers in the
$ABA$ triblock.
The horizontal direction shows the volume fraction $v_h$ of monomers
on the homopolymer chains in the system.
Each of these simulations (i.e.~each frame) is obtained running
the SCFT code on four processors and shows results on a $128 \times 128$
lattice ($32 R_g \times 32 R_g$).
}
\vspace{0.26cm}
\end{figure*}
%%%%%%%%%%%%%%%%%%%%%%%%%%%%%%%%%%%%%%%%%%%%%%%%%%%%%%%%%%%%%%%%




%%%%%%%%%%%%%%%%%%%%%%%%% FIGURE %%%%%%%%%%%%%%%%%%%%%%%%%%%%%%%
\begin{figure*}
\centerline
{
\epsfig{figure=chi_0.16_beta_0.5.epsi,angle=270,width=6in}
}
\caption
{
\label{fig_density_aba_a_chi_0.16}
%(fig density aba a chi 0.16)
Density configurations of symmetric ABA triblock / A homopolymer mixtures
for $\chi_{AB}$=$0.16$.
The data is organized in the same way as in Fig.~\protect{\ref{fig_density_aba_a_chi_0.10}}
Results for $f_A$=$0.30$, $v_h$=$0.20$ on a larger simulation lattice
are shown in Fig.~\protect{\ref{fig_big}}.
}
\vspace{0.26cm}
\end{figure*}
%%%%%%%%%%%%%%%%%%%%%%%%%%%%%%%%%%%%%%%%%%%%%%%%%%%%%%%%%%%%%%%%




%%%%%%%%%%%%%%%%%%%%%%%%% FIGURE %%%%%%%%%%%%%%%%%%%%%%%%%%%%%%%
\begin{figure*}
\centerline
{
\epsfig{figure=a_aba_1024x1024.epsi,angle=0,width=6in}
}
\caption
{
\label{fig_big}
%(fig big)
Density configuration of symmetric ABA triblock / A homopolymer mixture
for $f_A$=$0.30$, $v_h$=$0.20$ and $\chi_{AB}$=$0.16$.
The same parameters are used in one of the simulation results
shown in Fig.~\protect{\ref{fig_density_aba_a_chi_0.16}}
for a smaller $128 \times 128$ lattice.
The result shown here is for a $1024 \times 1024$ lattice
($256 R_g \times 256 R_g$).
}
\vspace{0.26cm}
\end{figure*}
%%%%%%%%%%%%%%%%%%%%%%%%%%%%%%%%%%%%%%%%%%%%%%%%%%%%%%%%%%%%%%%%


\end{document}